\documentclass{article}

\usepackage[T1]{fontenc}
\usepackage[polish]{babel}
\usepackage[utf8]{inputenc}

\usepackage[a4paper,top=2cm,bottom=2cm,left=2cm,right=2cm,marginparwidth=1.75cm]{geometry}

\usepackage{amsmath}
\usepackage{graphicx}
\usepackage[colorlinks=true, allcolors=blue]{hyperref}

\title{Symulacja spalania lasu trafionego piorunem i analiza optymalnego zagęszczenia drzew}
\author{Łukasz Minasiewicz, 286143}

\begin{document}
\maketitle
\section{Warunki Eksperymentu}
\subsection{Ogólne}
Eksperyment polega na przeprowadzeniu symulacji spalania lasu trafionego piorunem. Piorun trafia w losowe miejsce i podpala drzewo jeśli stoi ono w tym miejscu. Eksperyment ma na celu wykonać symulację setki tysięcy razy z różnymi parametrami aby znaleźć optymalny stopień zalesienia lda najwyższej jakości lasu.\\

\subsection{Parametry Symulacji}
\textbf{Długość i Szerokość Lasu:} Ilość pól wzdłuż i wszerz (które mogą zawierać lub nie zawierać drzew) na których przeprowadzamy symulację.\vspace{1ex}\\
\textbf{Stopień Zalesienia:} Procent pól w symulacji, które zawierają drzewa.\vspace{1ex}\\
\textbf{Wiatr:} Może być żaden, północny, południowy, wschodni, lub zachodni. Umożliwia rozprzestrzenianie się ognia o jedno pole dalej w daną stronę świata.\vspace{1ex}\\
\textbf{Wiek Drzew:} Parametr ten jest losowany dla każdego drzewa w każdej symulacji. Zmniejsza szansę zapłonu dla młodych drzew. Zakładamy, że w naturalnym lesie poszczególne drzewa mają różny wiek, oraz że strasze drzewa mają większą szansę na zapłon, ze względu na swoją wielkość oraz ilość suchej, martwej tkanki.\\

\subsection{Indeks Jakości Lasu (FQI) oraz Sposób Prowadzenia Eksperymentu}
Index Jakości Lasu (dalej FQI, ang. "Forest Quality Index") to indeks opracowany na potrzeby eksperymentu. Dla danej symulacji, indeks mierzy ilość pozostałych drzew po trafieniu piorunem i potencjalnym pożarze i mnoży ją przez wartość od $0.8$ do $1.2$ zależnie od procentu pozostałych drzew.\vspace{1ex}\\
Funkcja liniowa:

\[
Modyfikator(StopienPrzetrwania) = 0.8 + 0.4 * StopienPrzetrwania\\
\]

Zamysł FQI jest prosty - ważne jest dla nas przede wszystkim ile zostanie drzew na świecie. Ale zwracamy także uwagę na to, że lepiej jeśli spali się mniej drzew, niż więcej. Stąd w przypadku dwóch wyników z taką samą liczbą pozostałych drzew, wygra ta, w której pożar był mniejszy.\vspace{2ex}\\

Celem eksperymentu jest znalezienie optymalnego zalesienia dla danych parametrów. W tym celu symulację przeprowadzono 100 000 razy dla każdego stopnia zalesienia od 0\% do 100\% z krokiem o 0.5\%. Dla każdej partii 100 000 symulacji wyliczone i porównane zostały średnie wartości FQI. Wyniki eksperymentu przedstawione zostały poniżej.\\


\section{Wyniki Eksperymentu}
\subsection{Parametry Testowe}
Dla eksperymentu przyjęto poniższe parametry:\\
\begin{itemize}
  \item Rozmiary Lasu: $20x80$, $40x40$, $80x20$.
  \item Stopień Zalesienia: od $0$\% do $100$\%, z krokiem co $0.5$.\%
  \item Wiatr: 4 strony świata, oraz brak wiatru.
  \item Wielkość Próby: 100 000.
\end{itemize}\\

\subsection{Wyniki}

\begin{table}[h]
\centering
\begin{tabular}{|l|l|l|l|l|}\hline
Rozmiar & Wiatr & Wielkość Próby & Optymalne Zalesienie & FQI dla Optymalnego Zalesienia \\\hline
 $80x20$ & Brak & $100 000$ & $17$\% & \\\hline
 $80x20$ & Północ & $100 000$ &  & \\\hline
 $80x20$ & Południe & $100 000$ &  & \\\hline
 $80x20$ & Wschód & $100 000$ &  & \\\hline
 $80x20$ & Zachód & $100 000$ &  & \\\hline
 $40x40$ & Brak & $100 000$ &  & \\\hline
 $40x40$ & Północ & $100 000$ &  & \\\hline
 $40x40$ & Południe & $100 000$ &  & \\\hline
 $40x40$ & Wschód & $100 000$ &  & \\\hline
 $40x40$ & Zachód & $100 000$ &  & \\\hline
 $20x80$ & Brak & $100 000$ &  & \\\hline
 $20x80$ & Północ & $100 000$ &  & \\\hline
 $20x80$ & Południe & $100 000$ &  & \\\hline
 $20x80$ & Wschód & $100 000$ &  & \\\hline
 $20x80$ & Zachód & $100 000$ &  & \\\hline
\end{tabular}
\caption{\label{tab:h1} Optymalne zalesienie i FQI dla różnych parametrów.}
\end{table}\\

\section{Podsumowanie}

\\

\end{document}